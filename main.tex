\documentclass[11pt]{beamer}
\usetheme{IPL}
\usepackage[spanish]{babel}
\usepackage[utf8]{inputenc}
\usepackage[T1]{fontenc}
\usepackage{tikz}
\usetikzlibrary{backgrounds,mindmap}
\usetikzlibrary{arrows,positioning}
\usetikzlibrary{decorations.pathmorphing, decorations.pathreplacing, decorations.shapes} 
\usepackage{multicol}
\decimalpoint
%% Use any fonts you like.
\usepackage{times}
\setbeamercolor{Verde}{bg=COILVerde}
\setbeamercolor{Azul}{bg=IPLAzul}
\setbeamercolor{Gris}{bg=IPLGris}
\setbeamercolor{Orange}{bg=Naranja}
\setbeamercolor{Agua}{bg=aquamarine}
\setbeamercolor{Rojo}{bg=red}
\title{La modelación Matemática}
\subtitle{en la ingeniería}
\author{Abdul Abner Lugo Jiménez, PhD.}
\date{\today}
\institute{\url{alugo@ipl.edu.do}}

\begin{document}
\tikzstyle{every picture}+=[remember picture]
\everymath{\displaystyle}
\begin{frame}[plain,t]
\titlepage
\end{frame}

\begin{frame}{Iniciemos con una pregunta}
\onslide<1->{\fcolorbox{IPLAzul}{COILVerde}{?`Por qué un año dura $365$ días?}}\\
\onslide<2>{Sobre este sistema actúan muchos elementos, pero haremos mención sobre dos de ellos: la 
\fcolorbox{white}{red}{\textcolor{white}{\textbf{velocidad orbital}}} y la 
\fcolorbox{white}{IPLAzul}{\textcolor{white}{\textbf{velocidad tangencial}}}}
\end{frame}

\begin{frame}{Ecuaciones de las velocidades}
\begin{block}{Velocidad orbital y tangencial}
\begin{multicols}{2}
\begin{equation*}
V_{orbital}=\sqrt{\dfrac{G\cdot M_{s}}{r}}
\end{equation*}\pause
\begin{equation*}
V_{tangencial}=\omega\cdot r=\dfrac{2\pi}{T}\cdot r
\end{equation*}
\end{multicols}
\end{block}\pause
donde:
\begin{itemize}[<+-| alert@+>]
     \item $G$ es la constante gravitacional: $6.674\times10^{-11}\;N\frac{m^2}{kg^2}$
     \item $r$ es la distancia media entre la Tierra y el Sol: $1.496\times10^{11}\;m$
     \item $M_s$ es la masa del Sol: $1.989\times10^{30}\;kg$
     \item $T$ es el tiempo
\end{itemize}
\end{frame}

\begin{frame}{Calculemos el tiempo}
\begin{center}
{\fboxsep 12pt\fcolorbox {COILVerde}{COILVerde}{
\begin{minipage}[t]{4cm}
\begin{equation*}
\textcolor{white}{\pmb{T=2\pi\sqrt{\dfrac{r^3}{G\cdot M_s}}}}
\end{equation*}
\end{minipage}}}
\end{center}\pause

\begin{equation*}
\begin{split}
\textcolor{azulF}{\pmb{T=}}\;&\textcolor{azulF}{\pmb{2\pi\sqrt{\dfrac{(1.496\times10^{11}\;m)^3}{\left(6.674 
\times10^{-11}\;N\dfrac{m^2}{kg^2}\right)\cdot(1.989\times10^{30}\;kg)}}}}\\
\textcolor{azulF}{\pmb{\approx}}& \textcolor{azulF}{\pmb{31.554.894,53\;seg}}
\end{split}
\end{equation*}\pause
\begin{equation*}
\begin{split}
\textcolor{grisD}{\pmb{31.554.894,53\;seg}}&\textcolor{grisD}{\pmb{\left(\dfrac{1\;min}{60\;seg}\right)\left(\dfrac{
1\;hr}{60\;min}\right)\left(\dfrac{1\;\text{día}}{ 24\;hr}\right)}}\\
&\fcolorbox{red}{red}{\textcolor{white}{$\approx 365,242189\;\text{días}$}}
\end{split}
\end{equation*}

\end{frame}

\section{Modelación matemática}
\begin{frame}
\frametitle{Modelación matemática}
\framesubtitle{Introducción}
\begin{beamercolorbox}[wd={10cm},sep=1mm,left,rounded=true,shadow=true]{Azul}
\textcolor{white}{Las} \textcolor{red}{\textbf{leyes del universo}} \textcolor{white}{están escritas en el lenguaje de 
las matemáticas.} \pause \textcolor{white}{El álgebra es suficiente para resolver muchos problemas estáticos.}
\end{beamercolorbox}\pause

\begin{beamercolorbox}[wd={10cm},sep=1mm,left,rounded=true,shadow=true]{Orange}
\textcolor{white}{Es natural que las ecuaciones que involucran estos cambios se usen frecuentemente para describir el 
universo cambiante.} \pause \textcolor{white}{Cada una de estas ecuaciones que relaciona una función desconocida con 
una o más de sus derivadas la llamamos una \textcolor{red}{ecuación diferencial}.}
\end{beamercolorbox}
\end{frame}


\subsection{Definición}
\begin{frame}
\frametitle{Modelado matemático}
\framesubtitle{Definición}
El primer objetivo es lo que se conoce como \textcolor{red}{\textbf{modelado matemático}}, \pause el cual es crucial 
para formular una ecuación o sistema de ecuaciones que describa el problema físico que se quiere modelar.\\[2mm]\pause

El mismo involucra lo siguiente:
\begin{enumerate}[<+-| alert@+>]
    \item La formulación en términos matemáticos de un problema físico derivado de un fenómeno físico.
    \item La construcción de un modelo matemático.
\end{enumerate}
\end{frame}

\subsection{Algortimo de un MM}
\begin{frame}
\frametitle{Modelado Matemático}
\framesubtitle{Algorítmo de una modelado matemático}
\begin{figure}[H]
\centering
\begin{tikzpicture}[node distance=.2cm, auto]  
\tikzset{
    mynode/.style={rectangle,rounded corners,line width=1pt,draw=azulF, top color=IPLAzul, bottom color=IPLAzul, inner 
sep=1em, minimum size=2em, text centered},
    myarrow/.style={Naranja,->, >=latex', line width=2pt},
    mylabel/.style={text width=7em, text centered} 
}  
\node[mynode] (problema) {\textcolor{white}{Problema Físico}};\pause  
\node[below=3cm of problema] (dummy) {};
\node[mynode, left=of dummy] (modelo) {\textcolor{white}{Modelo Matemático}};
\draw[myarrow] (problema.west) -- ++(-.3,0) -- ++(0,-2) -|  (modelo.north);\pause
\node[mylabel, below left=of problema] (label1) {Formulación matemática}; \pause
\node[mynode, right=of dummy] (resultados) {\textcolor{white}{Resultados}};\pause
\draw[COILVerde,<->, >=latex',  line width=2pt, bend right=50, dashed] 
    (modelo.south) to node[auto, swap] {Análisis Matemático}(resultados.south);\pause
\draw[myarrow] (resultados.east) -- ++(.5,0) -- ++(0,2) -|  (problema.south);\pause
\node[mylabel, below right=of problema] (label2) {Interpretación};
\end{tikzpicture} 
\end{figure}
\end{frame}

\section{Ejemplos de modelado}
\subsection{Mecánica newtoniana}
\begin{frame}
\frametitle{Ejemplos de modelado}
\framesubtitle{Mecánica newtoniana}
\begin{figure}[H]
\begin{tikzpicture}
\draw[grisD,line width=1pt] (0,0.45) -- (8,0.45);
\draw[grisD,line width=1pt] (0,0.45) -- (0,4);
\draw[grisD!30,fill=grisD!30] (-.25,.20) rectangle (8,.44);
\draw[grisD!30,fill=grisD!30] (-.25,.20) rectangle (-.015,4);
\draw[red,line width=1pt] (4,0.5) -- (4,2.5) -- (6,2.5) -- (6,.5) -- cycle;
\draw[red] (5,1.5) node {$m$};
\draw[azulF,line width=1pt,->] (6,1.5) -- (7.5,1.5);
\draw[red,above] (6.75,1.5) node {$F(t)$};
\draw[azulF,line width=1pt] (5,2.5) -- (5,3);
\draw[azulF,line width=1pt,->] (5,3) -- (6.5,3);
\draw[red,above] (5.75,3) node {$x(t)$};
\draw[Verde,line width=1pt] (0,1.5) -- (.5,1.5);
\draw[decorate,decoration={coil,segment length=4pt},line width=1pt,Verde] (.5,1.5) -- (3.5,1.5);
\draw[Verde,line width=1pt] (3.5,1.5) -- (4,1.5);
\draw[red,above] (2,1.5) node {$k$};
\end{tikzpicture}
\end{figure}\pause

\begin{block}{Sistema masa-resorte}
\begin{equation}
F(t)=m\cdot\dfrac{d^2x(t)}{dt^2}+k\cdot x(t)
\end{equation}
\end{block}
\end{frame}

\section{Las matemáticas en la Ingeniería}
\subsection{?`Que ocurre?}
\begin{frame}
\frametitle{Las matemáticas en la Ingeniería}
\framesubtitle{?`Que ocurre?}
\begin{beamercolorbox}[wd={10cm},sep=1mm,rounded=true,shadow=true]{Orange}
Los estudiantes de ingeniería no ven la aplicación inmediata de la matemática;  \pause esto tiene su causa en la desarticulación entre los cursos básicos de matemáticas y los cursos subsiguientes.
\end{beamercolorbox}
\vspace{4mm}\pause

\begin{beamercolorbox}[wd={10cm},sep=1mm,rounded=true,shadow=true]{Verde}
Los cursos de matemáticas, y sus textos guía, no tienen como principal objetivo trabajar con cuestiones o sobre modelación matemática provenientes de situaciones reales, \pause sino
que tratan situaciones artificiales, diseñadas exclusivamente para el aula.
\end{beamercolorbox}
\end{frame}

\subsection{Beneficios de la modelación matemática}
\begin{frame}
\frametitle{Las matemáticas en la Ingeniería}
\framesubtitle{Beneficios de la modelación matemática}
La \textcolor{IPLAzul}{Modelación matemática} es tanto un dispositivo como un proceso académico que en el aula demuestra las siguientes ventajas:
\begin{enumerate}[<+-| alert@+>]
	\item Ayuda al estudiante a comprender mejor el escenario en el que se desarrolla.
	\item Refuerza el aprendizaje de las matemáticas (motivación).
	\item Estimula el desarrollo de algunas habilidades actitudinales de tipo matemático.
	\item Coadyuva a tener una mejor óptica de las matemáticas.
\end{enumerate}
\end{frame}

\ThankYouFrame

\end{document}
